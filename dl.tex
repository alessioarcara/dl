\documentclass{article}
% --- Layout e formattazione generale ---
\usepackage[a4paper, landscape, twocolumn, margin=1.5cm, columnsep=1cm]{geometry}
\usepackage[hidelinks]{hyperref}

% --- Matematica ---
\usepackage{amsthm}
\usepackage{amsmath}
\usepackage{amsfonts}
\usepackage{amssymb}
\usepackage{xcolor}
\usepackage{simplebnf}
\numberwithin{equation}{section}
\usepackage{tcolorbox}
\tcbuselibrary{theorems}
\usepackage{ebproof}

% --- Teoremi, definizioni e ambienti ---
\newenvironment{example}{
    \begin{tcolorbox}[colback=blue!5!white,colframe=blue!75!black,title=\bfseries Example]
}{
    \end{tcolorbox}
}
\newenvironment{exam}{
    \begin{tcolorbox}[colback=gray!5!white,colframe=gray!75!black,title=\bfseries
        Exam]
}{
    \end{tcolorbox}
}
\newtcbtheorem{Definition}{Definition}%
{colback=red!5!white,colframe=red!75!black,fonttitle=\bfseries}{}
\newtcbtheorem{Theorem}{Theorem}%
{colback=green!5!white,colframe=green!75!black,fonttitle=\bfseries}{}

% --- Grafica e disegni ---
\usepackage{graphicx}
\graphicspath{ {./images/} }
\usepackage{wrapfig}
\usepackage{neuralnetwork}


\author{Alessio Arcara}
\begin{document}
\begin{titlepage}
    \vspace*{\fill}
   \centering
   \Huge Languages and Algorithms for Artificial Intelligence\\
   \Large Alessio Arcara

   \mbox{}

   \begin{center}
       \includegraphics[width=0.4\linewidth]{prolog_meme}
   \end{center}
    \vspace*{\fill}
\end{titlepage}
\pagenumbering{gobble}

\tableofcontents
\clearpage

\pagenumbering{arabic}
\parindent 0pt

\cleardoublepage
\section{Course info}
Deep learning is a branch of machine learning, where we just use neural
networks and generally involves in use of many nested layers on them.

Usually them excel when we have a lot of data and a lot of features because
they exploit the so called deep feature of data, i.e automatic extraction of
other features from a preliminary set of given features and that can be
interesting if a feature can contribute very small but the combination of
multiple of them can be relevant, think for example if he wave an image with
some pixels. Maybe a pixel isn't so interesting, but the combination of many
pixels can denote maybe a border and that is the so called deep feature.

we do not have the theory of what is a span (algebra)
meta-knowledge 

belongs to a class of models
parametric description of a model 
optimization of parameters

we define a way to compare the models (loss function) to evaluate the results
of the models 

machine learning is just an optimization problem 
we just need to optimize the parameters and we choose the best one by the loss
function during the training

it is like to set a benchmark and automatize i 

- 1) parametric class of models 
- 2) define a way to compare the models (loss)
- 3) automatize it via optimization

why we are talking about learning 

we do not know the optimal solution in analytic form (closed form solution)
but we use iterative techniques, this iterations can be thought as progressive
learning. 

we need to thought the loss function on the parameters of the model 


3 different of learning:
- supervised learning (inputs+outputs) -> a mapping
 -- classification (discrete)
 -- regression (continuous)
- unsupervised learning (inputs) -> a distribution describing the data
- reinforcement learning (actions and rewards) -> behaviour of the agent
--  you have an agent interacting with an environment, it decides an action
(behaviour of the agent, we define them), an action modify the environment 
-- when you do an action you receive a reward 
-- you optimize the sum of cumulative rewards defined a behaviour of an agent.

-- discriminative techniques (in or outside of boundary)
-- generative techniques (you are interested in distribution of data)
-- component analysis (what features are important)

it's a zoo, a lot of techniques etc 

define a class of models 
define a loss function (what is a good distance to compare vectors, a good
notion of similarity between the vectors)

two probabilities distribution, how to compare these two distribution?
quadratic distance, but there are other distances.

this notion of similarity IS IMPORTANT, is never a trivial question 
recursive
SHORT INTRODUCTION OF NEURAL NETWORKS:
it is a network of artificial neurons

each neuron takes multiple inputs, it weights its inputs and combine them + 1
bias and then the weighted combination if passed to an activation function to
give an output.

the simplest activation function is logistic one 
sigma(wx+b)

there are different kind of activation functions
it is important to be a non linear function 

composing of linear functions is just a linear function and doesn't increase
the power of neural network, so it is important to use non linear functions.

binary threshold (1950 definition, is not a good function because of its
derivation)
logistic 
hyperbolic tangent -1 1 
relu

Dendrites (inputs)
Neuron weighted sum 
Axon binary threshold

PHOTO

human brains 
slow (chemical operation)

TOPOLOGY
types of neural networks


the semantic of feed-forward neural network is clear when you have cycles it
is a mess. to get its meaning we observe its output correlated with its
states.
brain has cycles in it, we do not know how cycles are structured.

transformers are feed-forward neural networks, 95\% of applications  use
feed-forward neural networks

feed-forward neural network single flow of information (from input to ouput)
recurrent

LAYERS 
in feed-forward networks, neurons are usually organized in layers. 
if there is more than one hidden layer is called shallow, or otherwise is deep

input 
hidden layer 
output layer 

DENSE layer (fully connected)
 each neuron at layer k-1 is connected to each neuron at k 
 one neuron 
 $$I^n\cdot W^n+B^1=O^1$$
 the operation can be vectorized to produce m outputs in parallel
 $$I^n\cdot W^{n\times m}+B^{m}=O^m$$
 just an algebric computation
 dense layers usually work on flat (unstructured inputs)
 the order of elements in a dense layer is completely irrilevant

 the problem of a dense layer is that you have an huge amount of parameters 
 1000 inputs 1000 outputs = 1million of parameters
CONVOLUTIONAL layers

each neuron at layer k-1 is connected to a fixed subset of neurons at layer k.
it is like the operation of convolution applying a kernel that is parametric
(learned during training)

the advantage is that we have less number of parameters (the kernel) that give
a feature map (the output)

FEATURES AND DEEP FEATURES 
Any individual measurable property of data useful for the solution of a
specific task is called a feature

In deep learning each layer synthesize new features in terms of the previous
ones in non linear elaborations.

the output of a hidden layer is so called deep features 
and that is the big difference between dl and other systems

KNOWLEDGE BASED SYSTEMS take an expert ask him to solve a problem and try to
mimic his approach by means of logical rules.
MACHINE LEARNING SYSTEMS ask the experts to get a good set of features
DEEP LEARNING get rid of the experts :)

just let the machine do the work for you 

in deep learning we have an encoding part 
and after decoding part, you use the information that you have extracted

1950 perceptron binary threshold activation function

ReLU and Dropout
Attention
BatchNormalization
Residual connections
YOLO v1: inceptionNet
DQN -> AlphaGo -> PPO -> Soft Actor Critic -> AlphaStar

when we are able to sample data from a learned distribution. 


paperswithcode to find interesting datasets for a particular task

\section{Example of successful applications}
IMAGE 

* classification 
* segmentation -> connected components (each pixels) -> separate also
different objects

\section{Introduction}

The generation is a continuous operation, a small movement correspond to a
tiny small modification

latent space interpolation -> move -> result 
between faces you lose quality

\paragraph{Perceptron.}

The perceptron: binary threshold activation function

\begin{figure}[!ht]
    \centering
    \begin{neuralnetwork}[height=4]
        \newcommand{\nodetextclear}[2]{}
        \newcommand{\nodetextx}[2]{$x_#2$}
        \newcommand{\nodetexty}[2]{$y_#2$}
        \inputlayer[count=3, bias=false, title=Input\\layer, text=\nodetextx]
        \hiddenlayer[count=4, bias=false, title=Hidden\\layer, text=\nodetextclear] \linklayers
        \outputlayer[count=2, title=Output\\layer, text=\nodetexty] \linklayers
    \end{neuralnetwork}
    \caption{Struttura di una rete neurale con uno strato nascosto} 
    \label{fig:neural_network}
\end{figure}

$$
\begin{aligned}
    \text{output}=\begin{cases}
        1 & \text{if}\displaystyle\sum_{i} w_ix_i+b\geq 0 \\ 
        0 & \text{otherwise}
    \end{cases}
    & 
    \text{output}=\begin{cases}
        1 & \text{if}\displaystyle\sum_{i} w_ix_i\geq -b \\ 
        0 & \text{otherwise}
    \end{cases}
\end{aligned}
$$

---- deadline 2/3 march
NOTES STRUCTURE:

- Reasons

- NEURAL NETWORKS 
   - PERCEPTRON
   - LAYERS ETC
   - examples

- Training
   - backpropagation
   - overfitting/underfitting
   - losses

- CNN

etc
----
\end{document}
